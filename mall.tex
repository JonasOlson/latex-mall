\documentclass[a4paper, article, oneside]{memoir}

\makeatletter
\def\@arabic#1{\number\numexpr#1-1\relax}
\makeatother

\usepackage{fontspec}
\usepackage{polyglossia}
\setdefaultlanguage{swedish}
\usepackage{datetime}
\renewcommand{\dateseparator}{-}
\usepackage[inline]{enumitem}

\usepackage[hyperref, table, dvipsnames, svgnames, x11names]{xcolor}

% Paketet svg möjliggör SVG-bilder, och kräver också några rader bugfix.
\usepackage{pdftexcmds}
\makeatletter
\let\pdfstrcmp\pdf@strcmp
\let\pdffilemoddate\pdf@filemoddate
\makeatother
\usepackage{svg}

% Gör att mellanslag i bildfilnamn fungerar.
\usepackage[space]{grffile}

% Matematik
\usepackage[intlimits]{mathtools}
\usepackage{amssymb}
\usepackage{mathrsfs}

% Kodlistning
\usepackage{minted}

\usepackage{hyperref}

\title{Rubrik}
\author{Författare}
%\date{\today\ (\currenttime)}

\hypersetup{
  pdftitle={\thetitle},
  pdfauthor={\theauthor}
}

\begin{document}
\maketitle

\chapter{Chapter}
\section{Section}
\section{Section}

\chapter{Chapter}
\section{Section}
\section{Section}

% Lista i löpande text
Kroppens fyra delar är
\begin{enumerate}
\item huvud,
\item axlar,
\item knä och
\item tå.
\end{enumerate}
En Renult består av
\begin{enumerate*}[label=(\alph*)]
\item fyra hjul och
\item en bult.
\end{enumerate*}

Newton säger oss att $\mathbf{F}_i = -(\mathbf{r}-\mathbf{r}_i)Gmm_i/|\mathbf{r}-\mathbf{r}_i|^3$ och därutöver att
\begin{align}
\mathbf{F} = \sum_{i=0}^{n-1} \mathbf{F}_i, \\
\mathbf{F} = m\mathbf{a}
\end{align}
där
\begin{align}
\mathbf{a} &= \dot{\mathbf{v}}, \nonumber \\
&= \ddot{\mathbf{r}}.
\end{align}

\begin{subequations}
\begin{align}
  f(x) =
  \begin{dcases*}
      \int_0^\infty \mathrm{e}^{-x}\,\mathrm{d}x & om $\int_0^\infty$ är positiv\\
    -x & annars
  \end{dcases*}
\end{align}

\begin{align}
  f(x) =
  \begin{dcases}
    \int_0^\infty, & \int_0^\infty > 0 \\
    x, & x < 0
  \end{dcases}
\end{align}

\begin{align}
  f(x) =
  \begin{cases}
    \int_0^\infty, & \int_0^\infty > 0 \\
    x, & x < 0
  \end{cases}
\end{align}
\end{subequations}

\begin{figure}[ht]
\centering
\includegraphics[width=0.9\textwidth]{bänk.jpeg}
\caption{Ad majorem\,--\,finstrukturkonstanten\,--\,Dei gloriam}
\label{fig:bänk}
\end{figure}

\begin{figure}[ht]
\centering
\includesvg[width=0.5\textwidth]{Smithdiagram}
\caption{Smithdiagram}
\label{fig:smith}
\end{figure}

\begin{table}[ht]  
\centering
\begin{tabular}{ll}
\toprule
veckodag & namn \\
\midrule
söndag & palmsöndagen \\
måndag & blåmåndagen \\
tisdag & vita tisdagen \\
onsdag & dymmelonsdag \\
torsdag & skärtorsdag \\
fredag & långfredagen \\
lördag & påskafton \\
(söndag) & (påskdagen) \\
\bottomrule
\end{tabular}
\caption{Stilla veckan}
\label{tab:stilla}
\end{table}

\appendix
\chapter{Chapter}

% Kod från fil
\inputminted[
mathescape=true, % Möjliggör matematiska uttryck i kommentarer
label=duff.c,
frame=lines,
framesep=2mm,
baselinestretch=1.2,
bgcolor=GhostWhite,
fontsize=\footnotesize,
linenos
]{c}{duff.c}

% Kod direkt i LaTeXdokumentet
\begin{minted}[bgcolor=Ivory]{python3}
[x**2 for x in range(10)]
\end{minted}
\end{document}
